\documentclass[a4paper]{article}
\usepackage[utf8]{inputenc}
\usepackage[english]{babel}
\usepackage{moreverb}
\usepackage{graphicx}
\usepackage{algorithm}% http://ctan.org/pkg/algorithm
\usepackage{algpseudocode}% http://ctan.org/pkg/algorithmicx
\title{Assignment 2 \\ EDA132 Applied Artificial Intelligence}
\date{\today}
\author{Fredrik Paulsson \\ dat11fp1@student.lu.se
\and Shan Senanayake \\ dat11sse@student.lu.se}
%\setcounter{secnumdepth}{5}
%\setcounter{tocdepth}{5}
\begin{document}
\maketitle
%\tableofcontents


\section{Introduction} In this assignment we were to produce two part solution
that estimates a robot's position on some grid by only using the output of a
noisy sensor. The two parts thus consist of a simulation of the robot and the
sensor as well as the algorithm that estimates the robot's position.

\section{Robot and Sensor}
Our robot and sensor are both implemented in the \texttt{MovingBot} class defined in \texttt{MovingBot.java}.

\subsection{Robot}
Our robot is implemented in such manner that for each call to the \texttt{move()} method the robot moves one step. The robot moves about in a grid consisting of 8x8 positions. The robot's movements are well defined according to a set of rules. Firstly the robot has a heading and with probability 0.7 the robot moves one step along the same heading. With probability 0.3 the robot moves one step along a new heading. If the robot faces a wall on the current heading a new heading is always chosen. Whenever a new heading is to be found it is chosen at uniform random. The start position of the robot on the grid is also chosen at uniform random.

We have chosen to define for headings for our bot, namely, north, south, east and west. This means that the robot only moves vertically and horizontally among the positions on the grid.

\subsection{Sensor}
The sensor is to be noisy and can therefore not output the exact position of the robot. Therefore another set of rules define what the sensor should output. For every call to the \texttt{sensorOutput()} function a point is given that has been calculated using the sensor's rules.

The rules that define the sensor output is the following. With probability 0.1 the true location of the robot is given as the output. With probability 0.4 any one of the eight neighbouring positions are chosen. With 0.4 probability any one of the sixteen positions surrounding the previous eight neighbouring positions are chosen. The remaining probability of 0.1 yields \emph{nothing} as the sensor output.

In those cases where the robot have some neighbouring positions outside of the grid and one of those positions is to be chosen as the sensor output we have chosen to give the output \emph{nothing}. This essentially means two things.  Firstly, there is a higher probability of producing the output \emph{nothing} if the robot is close to a wall than if it would be in the center of the grid. Secondly, the output \emph{nothing} may be thought of in two ways, an unclear reading or no reading at all as well as a position outside of the grid.




\section{Estimation Algorithm}
This part of the report explains how we estimated the position of the robot.
For the estimation we used a forward algorithm which will be explained in Section \ref{for_alg}. This algorithm is implemented in the \texttt{ForwardAlgorithm.java} file as well as the main model of the underlying markov chain. Before explaining the forward algorithm the model will have to be explained.

\subsection{Model}
To model the estimation, we have chosen to define a state (the state is implemented in the \texttt{State.java} file) which represents a point and a certain direction/heading. A state defines where the robot can potentially be, together all states define all the places the robot can be. Since we have 4 different directions, running on a 8x8 board gives us $4x8x8 = 132$ different possible states.

We chose to have our model as a matrix implementation. The states are represented in a vector called \texttt{stateProbability} with the size 132, the value for each entry is the likelyhood for the robot to be in that particular state. The transition model had to be represented in a matrix of the size of 132x132, where each entry transition[i][j] is the probability of transitioning from state i to state j.
The observation matrix is supposed to be represented as the observation in the diagonal, however we chose to represent is as a vector the same size as the \texttt{stateProbability} vector, since those are the observable states.


\subsection{Forward Algorithm}
\label{for_alg}
The forward algorithm uses a hidden markov model to calculate filtering or prediction depending on implementation. In out assignment we have used it as a filtering algorithm to calculate the most probable state the robot is in, given the evidence. The neat thing about the forward algorithm is that it takes $\mathcal{O}(1)$ time and space to calculate, it is however dependent on the amount of states available (which is defined by the model). 

\begin{algorithm}
  \caption{Forward algorithm}\label{forward}
  \begin{algorithmic}[1]
      \State \texttt{State Probability Vector S}
      \State \texttt{Transition model T}
      \State \texttt{Observation Vector O}
      \For{\texttt{row = 0 .. S.size}}
      	\For{\texttt{col = 0 .. S.size}}
      		\State $S(row)_{next} += \alpha O(row) \cdot T(col)(row) \cdot S(col)_{prev}$
      	\EndFor
      \EndFor
  \end{algorithmic}
\end{algorithm}

The above pseudocode (\ref{forward}) explains how we have implemented the forward algorithm. There are a few pointers that might require further explaination. $T(col)(row)$ is the transpose of the regular $T(col)(row)$ and it means the transition probability that we arrive to the current row from the given columns. $O(row)$ is the observation we made, ergo the probability that we got that certain sensor reading if we are in the state $row$. $S(col)$ is the current probability we have that we came from this state.

\section{Results and Discussion}
Originally we chose to force the sensor to output one of the neighbouring positions even if some of them were outside of the grid. Of course, we then made sure that the output was of course a position on the grid but this in turn meant that each of those positions had a higher probability of occurring. The probability of 0.4 may have been distributed on only three positions if the one of the eight neighbouring positions were to be chosen and the robot was in a corner of the grid. This caused some problems for us when we were to estimate the robots position as any one of these three positions had a higher probability of being output than the actual position. This made it difficult to estimate the true position of the robot on the grid. In those cases we could only estimate the true position of the robot in $35 - 40\%$ of the estimations.

\section{Source Code}
Our program is made up of six files which contain the source code. In this section we will briefly go through what each of these files contain.

\subsection{Point.java}
This file contains the class \texttt{Point} which we use to describe a position on the grid on which the robot moves. The class contains some trivial but very useful functions.
\subsection{State.java}
This file contains the class \texttt{State} which in our program means where the robot are and what heading it has. Thus, the \texttt{State} class maintains a \texttt{Point} denoting the position and an \texttt{int} to denote the heading. The class contains some trivial functions but also three nontrivial important functions. These are:
\begin{description}
\item{\texttt{isNeighbour}} This method checks if two states are neighbours. We have defined neighbour in such a way that two states are neighbours if the current and next positions are located (vertically or horizontally) one step from each other and if the current heading allows movement from the current position to the next position.
\item{\texttt{amountOfNeighbours}} This method simply returns how many positions the robot can move to that are located on the grid when it has the current state. Normally this method would return four as the robot can choose between four headings. However if the robot travels along the edge of the grid this will only be three and if the robot is located in a corner of the grid this will only be two.
\item{\texttt{encounterWall}} This method checks if the robot (in the current state) is positioned next to a wall and has a heading which face the wall. Essentially just checking if the robot were to move along it's current heading, would it go outside the grid.
\end{description}
\subsection{MovingBot.java}
This file contains the class \texttt{MovingBot} which composes our robot and sensor. There are three private functions in this class which are used to derive a new heading in those according to the rules specified in the assignment. There are also three public functions which are:
\begin{description}
\item{\texttt{move}} This method moves the robot one step according the rules specified in the assignment.
\item{\texttt{sensorOutput}} This method returns the sensor's reading of the robot's position. The output is created by following the specified rules defined in the assignment.
\item{\texttt{pos}} This method simply returns the true position of the robot. This used for us to visualize how much off the sensor and estimate positions are off by the real position.
\end{description}

\subsection{MovingBotGraphics.java}
This file contains the class \texttt{MovingBotGraphics} which sets up a Graphical User Interface (GUI) that visualizes the different positions on the grid. This class contains no interesting code but just the code that make up a GUI
\subsection{ForwardAlgorithm.java}
\subsection{BotMain.java}

\section{Running Instructions} The .java files containing the classes described
above are located in a package called othello placed on
/h/d9/v/dat11fp1/TAI/probabilistic\_reasoning on the student computer system. It
is simply to compile the java files in the package and then run BotMain from
outside that package. For example if the current directory is
/h/d9/v/dat11fp1/TAI the following command will compile the files: \texttt{javac
probabilistic\_reasoning/*.java}. After compiling the program can be run with
the following command: \texttt{java probabilistic\_reasoning.BotMain}.

%\begin{thebibliography}{1}
%\bibitem{wikipedia}
%http://en.wikipedia.org
%\end{thebibliography}
\end{document}
