\documentclass[a4paper]{article}
\usepackage[utf8]{inputenc}
\usepackage[english]{babel}
\usepackage{moreverb}
\usepackage{graphicx}
\title{Assignment 2 \\ EDA132 Applied Artificial Intelligence}
\date{\today}
\author{Fredrik Paulsson \\ dat11fp1@student.lu.se
\and Shan Senanayake \\ dat11sse@student.lu.se}
%\setcounter{secnumdepth}{5}
%\setcounter{tocdepth}{5}
\begin{document}
\maketitle
%\tableofcontents


\section{Introduction} In this assignment we were to produce two part solution
that estimates a robot's position on some grid by only using the output of a
noisy sensor. The two parts thus consist of a simulation of the robot and the
sensor as well as the algorithm that estimates the robot's position.

\section{Robot and Sensor}



\section{Estimation Algorithm}
This part of the report explains how we estimated the position of the robot. 
For the estimation we used a forward algorithm which will be explained in Section \ref{for_alg}. This algorithm is implemented in the \texttt{ForwardAlgorithm.java} file as well as the main model of the underlying markov chain. Before explaining the forward algorithm the model will have to be explained.

\subsection{Model}
To model the estimation, we have chosen to define a state (the state is implemented in the \texttt{State.java} file) which represents a point and a certain direction/heading. A state defines where the robot can potentially be, together all states define all the places the robot can be.   


\subsection{Forward Algorithm}
\label{for_alg}


\section{Results}


\section{Running Instructions} The .java files containing the classes described
above are located in a package called othello placed on
/h/d9/v/dat11fp1/TAI/probabilistic\_reasoning on the student computer system. It
is simply to compile the java files in the package and then run *******Main from
outside that package. For example if the current directory is
/h/d9/v/dat11fp1/TAI the following command will compile the files: \texttt{javac
probabilistic\_reasoning/*.java}. After compiling the program can be run with
the following command: \texttt{java probabilistic\_reasoning.*****Main}.

%\begin{thebibliography}{1}
%\bibitem{wikipedia}
%http://en.wikipedia.org
%\end{thebibliography}
\end{document}
