\documentclass[a4paper]{article}
\usepackage[utf8]{inputenc}
\usepackage[english]{babel}
\usepackage{moreverb}
\usepackage{graphicx}
%\usepackage{algorithm}% http://ctan.org/pkg/algorithm
%\usepackage{algpseudocode}% http://ctan.org/pkg/algorithmicx
\title{Assignment 3 \\ EDA132 Applied Artificial Intelligence}
\date{\today}
\author{Fredrik Paulsson \\ dat11fp1@student.lu.se
\and Shan Senanayake \\ dat11sse@student.lu.se}
%\setcounter{secnumdepth}{5}
%\setcounter{tocdepth}{5}
\begin{document}
\maketitle
%\tableofcontents


\section{Introduction}

\section{Running Instructions} The .java files containing the classes described
above are located in a package called othello placed on
/h/d9/v/dat11fp1/TAI/probabilistic\_reasoning on the student computer system. It
is simply to compile the java files in the package and then run BotMain from
outside that package. For example if the current directory is
/h/d9/v/dat11fp1/TAI the following command will compile the files: \texttt{javac
probabilistic\_reasoning/*.java}. After compiling the program can be run with
the following command: \texttt{java probabilistic\_reasoning.BotMain}. The program will run until a keyboard key is pressed and then will provide the final probability of the model and how many steps it has run as an output in the console. The wait time between iterations is set to 100 ms, however this can be changed by scolling with the mousewheel.

%\begin{thebibliography}{1}
%\bibitem{wikipedia}
%http://en.wikipedia.org
%\end{thebibliography}
\end{document}
