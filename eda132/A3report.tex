\documentclass[a4paper]{article}
\usepackage[utf8]{inputenc}
\usepackage[english]{babel}
\usepackage{moreverb}
\usepackage{graphicx}
%\usepackage{algorithm}% http://ctan.org/pkg/algorithm
%\usepackage{algpseudocode}% http://ctan.org/pkg/algorithmicx
\title{Assignment 3 \\ EDA132 Applied Artificial Intelligence}
\date{\today}
\author{Fredrik Paulsson \\ dat11fp1@student.lu.se
\and Shan Senanayake \\ dat11sse@student.lu.se}
%\setcounter{secnumdepth}{5}
%\setcounter{tocdepth}{5}
\begin{document}
\maketitle
%\tableofcontents


% Report
% The assignment must be documented in a report, which should contain the following:

% The name of the author, the title of the assignment, and any relevant information on the front page.
% A presentation of the assignment.
% A presentation of the improvement(s) you have chosen.
% A presentation of your implementation and how to run the executable.
% A print-out of the example set(s) and the resulting decision tree(s).
% Comments on the results you have achieved.

\section{Introduction}
In this assignment we had to implement an algorithm that creates a decision tree based on some examples. This tree can then be used in supervised learning agents. One decision tree is created for one relation. A relation consists of a name, attributes that can take certain values and examples which state the classification for different combinations of values for the attributes.

\section{Improvements}
We have chosen to implement two of the three given improvements. Firstly we have implemented the pruning improvement because it seems as the most useful improvement to make. Secondly we have implemented the algorithm in such a way that it is possible to assign real values on the attributes instead of just enumerating different values allowed.

The Weka ARFF format specifies several different types of values that each attribute can have. In our solution we only allow two types of values. These are numeric values (real and integral) and nominal values. Nominal values are an enumeration of values that the attribute may take.

\section{Results}

\section{Source Code}
Our implementation is split across several .java files. In this section we will briefly explain what the different files do and highlight the most interesting parts of each file.

\subsection{Attribute.java}
This class represents an attribute with a set of values. The values are represented as \texttt{String}-objects, which simply is the names of the values. In the case of numerical values the attribute should only have two given values, defined by a splitpoint. There are a bunch of methodes in this class however the most interesting ones for this assignment are:
\begin{description}
\item[\texttt{public boolean test(String attributeValue,Example example)}] This method takes a \texttt{String attributeValue} which represents a value from this attribute, and an \texttt{Example example} which represents a example that should be tested. 
This method tests if the value is numerical and that it belongs to the right value branch defined by \texttt{attributeValue}. If it is not a numerical attribute then it just checks that the example is the current branch being evaluated.
\item[\texttt{public void setSplitPoint(double splitPoint)}] This method takes a \texttt{double splitPoint} which defines what value the two branches should be spilt by. This method is only called if it is a numerical attribute and then overrides all the previous values in the value set.
\item[\texttt{public String getKeyIfNumerical(String value)}] This methos takes a \texttt{String value} which is a value from the set. If the attribute is numerical the correct \texttt{String}-object is returned. Otherwise the same value is returned. 
\end{description} 
\subsection{Goal.java}
\subsection{Example.java}
\subsection{Relation.java}
\subsection{DecisionTreeParser.java}
\subsection{DecisionNode.java}
This class is a simple interface, used to store nodes in a tree of two different types. Namely \texttt{AttributeNode} and \texttt{TerminalNode}. The only use for this is to be able to represent two types of classes, with some shared objects in the same tree-datastructure.
\subsection{AttributeNode.java}
\subsection{TerminalNode.java}
\subsection{DecisionTreeAlgorithm.java}
\subsection{DecTreeMain}

\section{Library}
We have chosen to use an external library to help get the $\chi^{2}$ table needed to do a $\chi^{2}$-pruning. 


\section{Running Instructions} The .java files containing the classes described
above are located in a package called decision\_tree placed on
/h/d9/v/dat11fp1/TAI/decision\_tree on the student computer system. It
is simply to compile the java files in the package and then run DecTreeMain from
outside that package. However, in order to compile the program the \texttt{ssj.jar} file needs to be included. For example if the current directory is
/h/d9/v/dat11fp1/TAI the following command will compile the files: \texttt{javac -cp "ssj.jar;"
decision\_tree/*.java}. After compiling the program can be run with
the following command: \texttt{java decision\_tree.DecTreeMain}. The program requires three arff files in order to run but these have been placed on the student computer system as well. The program will run according to the above instructions.

%\begin{thebibliography}{1}
%\bibitem{wikipedia}
%http://en.wikipedia.org
%\end{thebibliography}
\end{document}
